\def\wit3{{\scshape WIT$^{\bf 3}$}}

The FBK component of the system combination corresponds to the �contrastive XXX� system of the official FBK submission to the MT track.

The FBK system was built upon a standard phrase-based system using the Moses toolkit \cite{koehn07:moses}, and exploited the huge amount of parallel English-French and monolingual French training data provided by the organizers.




It featured a statistical log-linear model including a XXXX phrase translation model \cite{Bisazza:11b} and lexicalized reordering models (RMs), two French language models (LMs), as well as distortion, word and phrase penalties. 

In order to adapt the system on TED specific domain and genre, and to reduce the size of the system, data selection was carried out on all parallel English-French corpora, using the WIT3 training data as in-domain data.

Data selection was performed by means of XenC toolkit (\cite{}) exploiting the parallel xxx-modality (mode=3) and xxx-grams
separately for each available training corpora but the WIT3 data.

namely the data selection tool XenC \cite{Rousseau:2013:pbml}

It assumes the availability of a {\em seed} corpus, that in our case is  representative of the specific domain or project, and of a large generic corpus, from where to extract task-relevant sentences.  The XenC tools provides three modes to perform data selection:
\begin{description}
  \item[mode 1] implements a simple filtering process based on direct perplexity computation \cite{Gao:02}.
  \item[mode 2] monolingual cross-entropy difference~\cite{Moore:acl10}.
  \item[mode 3] bilingual cross-entropy difference \cite{axelrod-he-gao:2011:EMNLP}.
\end{description}

We used the modes 2 and 3. In the second mode each word sequence $s$ is scored with the difference of the cross-entropy computed on two LMs.
The first LM is estimated from the whole task-specific corpus, while the second LM is estimated from a random subset of the generic corpus, with a number of tokens similar to the specific one.


 different amount of data are selected ranging from 
for a total of 66M English running words.

Two TMs and two RMs were trained on WIT3 and selected data, separately, and combined using the fill-up (for TM) and backoff (for RM) techniques, using WIT3 as primary component.

The French side of WIT3 and selected data were employed to estimate a mixture language model [58]. A second huge French LM was estimated on the monolingual French available data of about 2.4G running words. 

Both LMs have order five and were smoothed by means of the interpolated Improved Kneser-Ney method [59]; the second LM was also pruned-out of singleton n-gram (n>2).

Tuning of the system was performed on dev2010 by optimizing BLEU using Minimum Error Rate Training [20]. It is worth noticing that the dev2010 and test2010 data were added to the training data in order to build the system actually employed.